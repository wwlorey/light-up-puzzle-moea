\documentclass[11pt]{article}
\usepackage[left=3cm, right=3cm, top=2cm]{geometry}
\usepackage{graphicx}

% Prevent heading numbers
\setcounter{secnumdepth}{0}

% Command definitions 
\newcommand{\fitnessplotcaption}[1]{\caption{Evaluations versus Average Local Subfitness and Evaluations versus 
    Local Best Subfitness for the \textbf{{#1}}, Averaged Over All Runs}}

\newcommand{\addgraphic}[1]{\centerline{\includegraphics[scale=0.6]{output/#1__graph.png}}}

\newcommand{\tablecaption}[1]{\caption{Statistical Analysis performed on the {#1}}}

\title{CS5401 FS2018 Assignment 1d}
\author{William  Lorey \\ wwlytc@mst.edu}
\date{}

\begin{document}

\maketitle

\tableofcontents

\section{Introduction}

Assignment 1d involved implementing a Multi-Objective Evolutionary Algorithm (MOEA)
to more effectively solve Light Up puzzles by balancing the fulfillment of three
objectives: 

\begin{enumerate}

    \item maximize the number of cells lit up (represented in this implementation
    as a ratio of lit cells to the total number of white cells)

    \item minimize the number of bulbs shining on each other

    \item minimize the number of black cell adjacency constraint violations

\end{enumerate}

For BONUS \#1, a fourth objective was added, namely minimizing the number of bulbs
placed on the board.

This report outlines this solution's particular
implementation of a MOEA, the impact of initialization strategies on the MOEA's 
performance, a comparison between parent selection, survival strategy, and survival
selection strategies on MOEA performance, as well as the impact of increasing the
number of objectives on non-domination and MOEA performance (BONUS \#1).


\section{MOEA Overview}

The MOEA implemented in this assignment is based on the NSGA-II algorithm. It begins,
similar to a standard evolutionary algorithm, by creating an initial population using
either uniform random or validity enforced plus uniform random initialization, the 
settings for which are specified in the algorithm configuration file. That population
is evaluated and the subfitnesses are determined and assigned to each individual 
in the population.

The population is then evaluated on the basis of non-domination. A list of Pareto fronts is created
from the initial population where all genotypes in a given front are not dominated
by any other genotypes in that front while genotypes in higher level fronts are dominated
by genotypes in lower level fronts. The 'best' genotypes, those in the best level
of non-domination, are assigned to level number one. Subsequent levels increase in
increments of one for other levels of non-domination.

The fitness of each genotype is then set to its level in the list of Pareto fronts, with
individuals exhibiting a smaller fitness (level number) are more fit. A binary tournament selection 
is performed to choose breeding parents. Then offspring are created using an n-point crossover
recombination (with n determined in the configuration file). Following that, mutation is performed,
completing the child population.

For the standard NSGA-II configuration (exhibited in the deliverables configuration folder),
the plus survival strategy is exhibited, combining the children and parent populations into 
one large population from which to choose the new population. Individuals are then selected for 
survival using a binary tournament selection and the process is repeated using the new population
until the end of the experiment.


\section{Impact of Initialization on MOEA Performance}

The effect of Validity Enforced plus Uniform Random versus Uniform Random initialization
was examined in this experiment for both the provided puzzle and randomly generated puzzles.
One would assume that an initialization method utilizing Validity
Enforced initialization would outperform a solely Uniform Random initialization. After performing 
statistical analysis on the experiment data, it was concluded that there is in fact no tangible difference
between the initialization methods for the tested puzzles. Table \ref{init_random} and Table \ref{init_provided} 
each display statistical analysis supporting this finding. 

This statistical analysis for this section was performed on the average of all last best subfitnesses, giving thirty
data points to compare for each MOEA initialization method. The analysis consisted of 
performing an f-test, which determined if variances could be
treated as equal. In both cases, the f-test yielded that unequal variances should be assumed. Following
 the f-test, the two-tailed t-test was performed assuming unequal variances. This test yielded
(in both cases) that neither initialization method was statistically better for the set of Light Up
puzzles tested.

To visually interpret the data, plots of evaluations versus average local subfitness and evaluations 
versus local best subfitness were graphed for each of the subfitnesses collected in this experiment (not
including the bonus):
ratio of lit cells to total number of white cells, number of bulbs shining on each other, and 
number of black cell constraints not met. For concision, figures pertaining to only the provided puzzle
are discussed in this section as the randomly generated puzzle results are quite similar. Figures 
\ref{fig:website_v_ratio}, \ref{fig:website_v_shine}, \ref{fig:website_v_black}, 
\ref{fig:website_u_ratio}, \ref{fig:website_u_shine}, and \ref{fig:website_u_black} depict experiment 
plots for the provided puzzle while figures
\ref{fig:random_gen_v_ratio}, \ref{fig:random_gen_v_shine}, \ref{fig:random_gen_v_black},
\ref{fig:random_gen_u_ratio}, \ref{fig:random_gen_u_shine}, and \ref{fig:random_gen_u_black} depict 
experiment plots for the randomly generated puzzle. 

Figure \ref{fig:website_v_ratio} and Figure \ref{fig:website_u_ratio} depict the lit cell ratio subfitness
plots for the Validity Enforced plus Uniform Random and Uniform Random initialized experiments,
respectively. In the uniform random case (Figure \ref{fig:website_u_ratio}), both the average local
list cell ratio and the local best lit cell ratio started lower than those of the validity enforced 
experiment. However, as the experiments progressed, both the average and best subfitnesses of each
experiment reached appreciatively the same point without the best subfitness plateauing. This implies
that letting the experiment run for longer would produce a more fit solution, with respect to
the lit cell ratio. Note that the lit cell ratio subfitness, the metric was maximized.

Figure \ref{fig:website_v_shine} and Figure \ref{fig:website_u_shine} depict the evaluations versus
subfitness plots for the bulb shine constraint violations. These plots behaved quite similarly to the
lit cell ratio plots in that the Validity Enforced plus Uniform Random plot started with a higher average
number of bulb shine constraints violated while the plain Uniform Random plot had a lower number of
initial constraint violations. This is logical as enforcing validity has the potential to place more bulbs,
which creates opportunity for more bulbs
to shine on each other, further adding to the number of bulb shine violations. The local best for both
plots stayed right at zero bulb constraint violations, implying that most, if not all, experiments always had
at least one member of the population with no bulb shine constraints. Because this subfitness is to be
minimized as part of the MOEA, it is peculiar that the average number of bulb constraints increases as 
the experiments continue. This implies that as more bulbs are placed on the board, the multi-objective
nature of the algorithm allows for more and more bulbs to shine on each other, keeping those individuals
with more bulb-shine in the population so long as the levels of non-domination dictate it.

The third subfitness, black cell constraint violations, was examined for both initialization
schemes in Figure \ref{fig:website_v_black} and Figure \ref{fig:website_u_black}. Before examining
these plots, it was hypothesized that the Validity Enforced plus Uniform Random initialization 
method would be superior with respect to minimizing the black cell constraint violations when compared
to the Uniform Random only initialization. While the statistical analysis was not granular enough to
definitively prove this, the graphs provide anecdotal evidence that the Validity Enforced plus Uniform
Random initialization method is in fact superior when it comes to the black cell constraint violations.
This conclusion was drawn because the Validity Enforced plus Uniform Random method had lower average and
local best black cell constraints violated at each point on the graph, across all experiments.


\section{Comparison of Parent Selection Strategy, Survival Strategy, and Survival Selection Strategy
MOEA Configurations}

All combinations of the following MOEA configurations were compared, for both randomly generated 
puzzles and the provided puzzle, to determine which configuration combination was optimal.

\begin{itemize}
    \item Parent Selection
    \begin{itemize}
        \item Fitness Proportional
        \item Binary Tournament
    \end{itemize}

    \item Survival Selection
    \begin{itemize}
        \item Fitness Proportional
        \item Binary Tournament
    \end{itemize}

    \item Survival Strategy 
    \begin{itemize}
        \item Plus
        \item Comma
    \end{itemize}
\end{itemize}

Each combination of configurations resulted in \begin{math} n = 2^3 = 8 \end{math} distinct configurations. Each configuration
was tested against each other configuration (excluding itself), resulting in  

\[ n * (n - 1) = 8 * 7 = 56 \]

comparisons per puzzle type. 56 comparisons were performed for both the provided puzzle and the randomly generated puzzle, 
resulting in 112 comparisons total.

The explicit statistical analysis for each individual comparison is not tabulated in this report for brevity, however, the statistical analysis
methodology is described as follows: First, the last best subfitness for each of the three subfitness were aggregated and written to separate 
files for each of the configuration schemes listed above. This resulted in \begin{math} 3 * 8 = 24 \end{math} data files
to drive the statistical analysis for each the provided puzzle and the randomly generated puzzle (48 in total).
Following this, each possible pairing (in which order does matter) was performed, 
comparing from each configuration each of the three last best subfitness lists using first an f-test to determine whether or not equal
variances could be assumed followed by a two-tailed t-test assuming either equal or unequal variances (depending on the 
outcome of the f-test). The f-test would then yield if a configuration was better with respect to each subfitness.
If the number of a configuration's subfitness' superiorities outnumbered the number of a configuration's subfitness'
subordinates for a given comparison, that configuration was marked as being better than the other, incrementing a 
dominance count associated with that configuration. The configuration(s) with the highest dominance count are statistically
better than all other configurations tested for the given test puzzles.

The randomly generated puzzle configurations yielded two configurations that were both 
equally as optimal (they both dominated two other configurations, while all other configurations
dominated no other configurations). These were MOEAs configured 
with (1) fitness proportional parent selection, tournament survival selection, 
and plus survival strategy and (2) tournament parent
selection, tournament survival selection, and plus survival strategy. 
Both of these solutions each were dominant over two other configurations
out of all the other configurations. They were either at least as good as or 
subordinate to the other configurations.
The outcome of these comparisons proved that for this MOEA tested against randomly generated 
puzzles, the plus method for survival strategy is optimal, as both dominant 
configurations employed this strategy for at least one selection. The results also showed 
that relatively \textit{low} selection pressure for parent and survival
selections led to better algorithm results. The binary tournament selection applies low selection
pressure and is prevalent in both configurations - the first configuration used binary tournament 
selection for survival selection while the second used a binary tournament selection for 
both the parent 
and survival selections. This method of selection favors exploration over exploitation in
traversing the solution search space for optimal bulb placements.

%        ('../output/random_gen/random_gen__fitness_proportional_parent__tournament_survival__plus__last_best_local_black_constrs.txt',

%        ('../output/random_gen/random_gen__tournament_parent__tournament_survival__plus__last_best_local_black_constrs.txt',

The provided puzzle configurations yielded one configuration that was optimal when compared 
to all other solutions. In fact,
this configuration was the only solution found to be statistically better than all other 
configurations. This configuration involved
a tournament parent selection and tournament survival selection with a plus survival strategy.
Similarly to the above dominance results, the results for the provided puzzle showed that 
configurations employing a low selection pressure and a plus survival strategy produced better 
results. This dominance was more prevalent for the provided puzzle, as the winning configuration
dominated all other configurations. Again, this proves that for this particular MOEA, an approach
of exploration over exploitation is integral in achieving more optimal results.

%        ('../output/website_puzzle/website_puzzle__tournament_parent__tournament_survival__plus__last_best_local_black_constrs.txt',


\section{BONUS \#1: Impact of Increasing Number of Objectives on MOEA Performance}

For the first bonus, a fourth subfitness was added to the MOEA, namely a constraint minimizing
the number of bulbs placed on the board. This constraint directly contradicted one of the main
measures of an EA's performance - the ratio of lit cells to the total number of white cells.
For an MOEA to be successful with only the lit cell ratio, bulb shine constraint violations, and
black cell constraint violations subfitnesses, it did not explicitly need to minimize the number 
of bulbs
placed. In fact, the only thing \textit{limiting} the number of bulbs placed on the board was the
bulb shine constraint violation metric. By adding the fourth subfitness of minimizing the number
of placed bulbs, the MOEA is more likely to be conservative regarding the placement of bulbs on
a board, as now half of the four total subfitness metrics actively limit the number of placed bulbs.
Although statistical analysis was not performed regarding the implication of an increased number
of subfitness objectives and its impact on MOEA performance, the following hypothesis is offered
instead: If a fourth subfitness objective is added which minimizes the number of bulbs placed
on the board, the best level of non-domination will (overall, across many runs) prove to contain 
genotypes with lower lit-cell ratios as the incentive for placing more bulbs will be limited
while the incentive for lighting up many cells will stay the same.


\section{Appendix: Figures and Tables}

\textit{Note that figures and tables found in this section are referenced above in the body of 
the document.}


% Validity Enforced plus Uniform Random Init. Figures

\begin{figure}
    \addgraphic{website_puzzle/website_puzzle_lit_cell_ratio}
    \fitnessplotcaption{Lit Cell Ratio Subfitness, Validity Enforced plus Uniform Random Initialized, Provided Puzzle}
    \label{fig:website_v_ratio}
\end{figure}

\begin{figure}
    \addgraphic{website_puzzle/website_puzzle_bulb_shine_constr}
    \fitnessplotcaption{Bulb Shine Constraint Subfitness, Validity Enforced plus Uniform Random Initialized, Provided Puzzle}
    \label{fig:website_v_shine}
\end{figure}

\begin{figure}
    \addgraphic{website_puzzle/website_puzzle_black_cell_constr}
    \fitnessplotcaption{Black Cell Constraint Subfitness, Validity Enforced plus Uniform Random Initialized, Provided Puzzle}
    \label{fig:website_v_black}
\end{figure}

\begin{figure}
    \addgraphic{random_gen/random_gen_lit_cell_ratio}
    \fitnessplotcaption{Lit Cell Ratio Subfitness, Validity Enforced plus Uniform Random Initialized, Randomly Generated Puzzle}
    \label{fig:random_gen_v_ratio}
\end{figure}

\begin{figure}
    \addgraphic{random_gen/random_gen_bulb_shine_constr}
    \fitnessplotcaption{Bulb Shine Constraint Subfitness, Validity Enforced plus Uniform Random Initialized, Randomly Generated Puzzle}
    \label{fig:random_gen_v_shine}
\end{figure}

\begin{figure}
    \addgraphic{random_gen/random_gen_black_cell_constr}
    \fitnessplotcaption{Black Cell Constraint Subfitness, Validity Enforced plus Uniform Random Initialized, Randomly Generated Puzzle}
    \label{fig:random_gen_v_black}
\end{figure}


% Uniform Random Init. Figures

\begin{figure}
    \addgraphic{website_puzzle/website_puzzle_uniform_random_init_lit_cell_ratio}
    \fitnessplotcaption{Lit Cell Ratio Subfitness, Uniform Random Initialized, Provided Puzzle}
    \label{fig:website_u_ratio}
\end{figure}

\begin{figure}
    \addgraphic{website_puzzle/website_puzzle_uniform_random_init_bulb_shine_constr}
    \fitnessplotcaption{Bulb Shine Constraint Subfitness, Uniform Random Initialized, Provided Puzzle}
    \label{fig:website_u_shine}
\end{figure}

\begin{figure}
    \addgraphic{website_puzzle/website_puzzle_uniform_random_init_black_cell_constr}
    \fitnessplotcaption{Black Cell Constraint Subfitness, Uniform Random Initialized, Provided Puzzle}
    \label{fig:website_u_black}
\end{figure}

\begin{figure}
    \addgraphic{random_gen/random_gen_uniform_random_init_lit_cell_ratio}
    \fitnessplotcaption{Lit Cell Ratio Subfitness, Uniform Random Initialized, Randomly Generated Puzzle}
    \label{fig:random_gen_u_ratio}
\end{figure}

\begin{figure}
    \addgraphic{random_gen/random_gen_uniform_random_init_bulb_shine_constr}
    \fitnessplotcaption{Bulb Shine Constraint Subfitness, Uniform Random Initialized, Randomly Generated Puzzle}
    \label{fig:random_gen_u_shine}
\end{figure}

\begin{figure}
    \addgraphic{random_gen/random_gen_uniform_random_init_black_cell_constr}
    \fitnessplotcaption{Black Cell Constraint Subfitness, Uniform Random Initialized, Randomly Generated Puzzle}
    \label{fig:random_gen_u_black}
\end{figure}


% Other EA Config Comparison Figures

%\begin{figure}
%    \addgraphic{random_gen/random_gen}
%    \fitnessplotcaption{}
%    \label{fig:}
%\end{figure}


% Statistical Analysis Tables

\begin{table}[] 
\tablecaption{Uniform Random and Validity Enforced Uniform Random Initialized, Randomly Generated Puzzle, EA configurations}        
\label{init_random}                 
\resizebox{\textwidth}{!}{%        
\begin{tabular}{|l|l|l|}           
\hline               
  & random\_gen & random\_gen\_uniform\_random\_init  \\ \hline 
 mean & 1.10325284795678 & 1.0764116503308314 \\ \hline 
 variance & 0.04970344216568983 & 0.04355857958196582 \\ \hline 
 standard deviation & 0.22294268807406498 & 0.20870692269775296 \\ \hline 
 observations & 30 & 30 \\ \hline 
 df & 29 & 29 \\ \hline 
 F & 1.141071234248146 &   \\ \hline 
 F critical & 0.5373999648406917 &   \\ \hline 
 Unequal variances assumed &  &    \\ \hline 
  &  &     \\ \hline 
  observations & 30 &  \\ \hline 
  df & 31 &  \\ \hline 
  t Stat & 0.47331304656977363 &  \\ \hline 
  P two-tail & 0.6377741699825987 &  \\ \hline 
  t Critical two-tail & 2.0395 &  \\ \hline 
  Nether random\_gen\_uniform\_random\_init nor & & \\
 random\_gen is statistically better &  &   \\ \hline 
  \end{tabular}%     
}                    
\end{table}

\begin{table}[] 
\tablecaption{Uniform Random and Validity Enforced Uniform Random Initialized, Provided Puzzle, EA configurations}        
\label{init_provided}                 
\resizebox{\textwidth}{!}{%        
\begin{tabular}{|l|l|l|}           
\hline               
  & website\_puzzle & website\_puzzle\_uniform\_random\_init  \\ \hline 
 mean & 0.8031737242867948 & 0.7959489651519909 \\ \hline 
 variance & 0.000603666878279215 & 0.00058947613524421 \\ \hline 
 standard deviation & 0.024569633254878164 & 0.02427912962287178 \\ \hline 
 observations & 30 & 30 \\ \hline 
 df & 29 & 29 \\ \hline 
 F & 1.02407348183676 &   \\ \hline 
 F critical & 0.5373999648406917 &   \\ \hline 
 Unequal variances assumed &  &    \\ \hline 
  &  &     \\ \hline 
  observations & 30 &  \\ \hline 
  df & 31 &  \\ \hline 
  t Stat & 1.1263571505364551 &  \\ \hline 
  P two-tail & 0.26465388827990055 &  \\ \hline 
  t Critical two-tail & 2.0395 &  \\ \hline 
  Nether website\_puzzle\_uniform\_random\_init nor & & \\
 website\_puzzle is statistically better &  &   \\ \hline 
  \end{tabular}%     
}                    
\end{table}

\end{document}

