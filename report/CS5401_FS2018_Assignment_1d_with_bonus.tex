\documentclass[11pt]{article}
\usepackage[left=3cm, right=3cm, top=2cm]{geometry}
\usepackage{graphicx}

% Prevent heading numbers
\setcounter{secnumdepth}{0}

% Command definitions 
\newcommand{\fitnessplotcaption}[1]{\caption{Evaluations versus Average Local Subfitness and Evaluations versus 
    Local Best Subfitness for the \textbf{{#1}}, Averaged Over All Runs}}

\newcommand{\addgraphic}[1]{\centerline{\includegraphics[scale=0.6]{output/#1__graph.png}}}

\newcommand{\tablecaption}[1]{\caption{Statistical Analysis performed on the {#1}}}

\title{CS5401 FS2018 Assignment 1d}
\author{William  Lorey \\ wwlytc@mst.edu}
\date{}

\begin{document}

\maketitle

\tableofcontents

\section{Introduction}

Assignment 1d involved implementing a Multi-Objective Evolutionary Algorithm (MOEA)
to more effectively solve Light Up puzzles by balancing the fulfillment of three
objectives: 

\begin{enumerate}

\item maximize the number of cells lit up (represented in this implementation
as a ratio of lit cells to the total number of white cells)

\item minimize the number of bulbs shining on each other

\item minimize the number of black cell adjacency constraint violations

\end{enumerate}

For BONUS \#1, a fourth objective was added, namely minimizing the number of bulbs
placed on the board.

This report outlines this solution's particular
implementation of a MOEA, the impact of initialization strategies on the MOEA's 
performance, a comparison between parent selection, survival strategy, and survival
selection strategies on MOEA performance, as well as the impact of increasing the
number of objectives on non-domination and MOEA performance (BONUS \#1).


\section{MOEA Overview}

The MOEA implemented in this assignment is based on the NSGA-II algorithm. It begins,
similar to a standard evolutionary algorithm, by creating an initial population using
either uniform random or validity enforced plus uniform random initialization, the 
settings for which are specified in the algorithm configuration file. That population
is evaluated and the subfitnesses are determined and assigned to each individual 
in the population.

The population is then evaluated on the basis of non-domination. A list of Pareto fronts is created
from the initial population where all genotypes in a given front are not dominated
by any other genotypes in that front while genotypes in higher level fronts are dominated
by genotypes in lower level fronts. The 'best' genotypes, those in the best level
of non-domination, are assigned to level number one. Subsequent levels increase in
increments of one for other levels of non-domination.

The fitness of each genotype is then set to its level in the list of Pareto fronts, with
individuals exhibiting a smaller fitness (level number) are more fit. A binary tournament selection 
is performed to choose breeding parents. Then offspring are created using an n-point crossover
recombination (with n determined in the configuration file). Following that, mutation is performed,
completing the child population.

For the standard NSGA-II configuration (exhibited in the deliverables configuration folder),
the plus survival strategy is exhibited, combining the children and parent populations into 
one large population from which to choose the new population. Individuals are then selected for 
survival using a binary tournament selection and the process is repeated using the new population
until the end of the experiment.


\section{Impact of Initialization on MOEA Performance}

The effect of Validity Enforced plus Uniform Random versus Uniform Random initialization
was examined in this experiment for both the provided puzzle and randomly generated puzzles.
One would assume that an initialization method utilizing Validity
Enforced initialization would outperform a solely Uniform Random initialization. After performing 
statistical analysis on the experiment data, it was concluded that there is in fact no tangible difference
between the initialization methods for the tested puzzles. Table \ref{init_random} and Table \ref{init_provided} 
each display statistical analysis supporting this finding. 

This statistical analysis consisted of performing an f-test, which determined if variances could be
treated as equal. In both cases, the f-test yielded that unequal variances should be assumed. Following
 the f-test, the two-tailed t-test was performed assuming unequal variances. This test yielded
(in both cases) that neither initialization method was statistically better for the set of Light Up
puzzles tested.

To visually interpret the data, plots of evaluations versus average local subfitness and evaluations 
versus local best subfitness were graphed for each of the subfitnesses collected in this experiment (not
including the bonus):
ratio of lit cells to total number of white cells, number of bulbs shining on each other, and 
number of black cell constraints not met. For concision, figures pertaining to only the provided puzzle
are discussed in this section as the randomly generated puzzle results are quite similar. Figures 
\ref{fig:website_v_ratio}, \ref{fig:website_v_shine}, \ref{fig:website_v_black}, 
\ref{fig:website_u_ratio}, \ref{fig:website_u_shine}, and \ref{fig:website_u_black} depict experiment 
plots for the provided puzzle while figures
\ref{fig:random_gen_v_ratio}, \ref{fig:random_gen_v_shine}, \ref{fig:random_gen_v_black},
\ref{fig:random_gen_u_ratio}, \ref{fig:random_gen_u_shine}, and \ref{fig:random_gen_u_black} depict 
experiment plots for the randomly generated puzzle. 

Figure \ref{fig:website_v_ratio} and Figure \ref{fig:website_u_ratio} depict the lit cell ratio subfitness
plots for the Validity Enforced plus Uniform Random and Uniform Random initialized experiments,
respectively. In the uniform random case (Figure \ref{fig:website_u_ratio}), both the average local
list cell ratio and the local best lit cell ratio started lower than those of the validity enforced 
experiment. However, as the experiments progressed, both the average and best subfitnesses of each
experiment reached appreciatively the same point without the best subfitness plateauing. This implies
that letting the experiment run for longer would produce a more fit solution, with respect to
the lit cell ratio. Note that the lit cell ratio subfitness, the metric was maximized.

Figure \ref{fig:website_v_shine} and Figure \ref{fig:website_u_shine} depict the evaluations versus
subfitness plots for the bulb shine constraint violations. These plots behaved quite similarly to the
lit cell ratio plots in that the Validity Enforced plus Uniform Random plot started with a higher average
number of bulb shine constraints violated while the plain Uniform Random plot had a lower number of
initial constraint violations. This is logical as enforcing validity has the potential to place more bulbs,
which creates opportunity for more bulbs
to shine on each other, further adding to the number of bulb shine violations. The local best for both
plots stayed right at zero bulb constraint violations, implying that most, if not all, experiments always had
at least one member of the population with no bulb shine constraints. Because this subfitness is to be
minimized as part of the MOEA, it is peculiar that the average number of bulb constraints increases as 
the experiments continue. This implies that as more bulbs are placed on the board, the multi-objective
nature of the algorithm allows for more and more bulbs to shine on each other, keeping those individuals
with more bulb-shine in the population so long as the levels of non-domination dictate it.

The third subfitness, black cell constraint violations, was examined for both initialization
schemes in Figure \ref{fig:website_v_black} and Figure \ref{fig:website_u_black}. Before examining
these plots, it was hypothesized that the Validity Enforced plus Uniform Random initialization 
method would be superior with respect to minimizing the black cell constraint violations when compared
to the Uniform Random only initialization. While the statistical analysis was not granular enough to
definitively prove this, the graphs provide anecdotal evidence that the Validity Enforced plus Uniform
Random initialization method is in fact superior when it comes to the black cell constraint violations.
This conclusion was drawn because the Validity Enforced plus Uniform Random method had lower average and
local best black cell constraints violated at each point on the graph, across all experiments.


\section{Comparison of Parent Selection, Survival Strategy, and Survival Selection Strategies}

The following MOEA configurations were compared to determine which combination was optimal.

\begin{itemize}

    \item Parent Selection
    \begin{itemize}
        \item Fitness Proportional
        \item Binary Tournament
    \end{itemize}

    \item Survival Selection
    \begin{itemize}
        \item Fitness Proportional
        \item Binary Tournament
    \end{itemize}

    \item Survival Strategy 
    \begin{itemize}
        \item Plus
        \item Comma
    \end{itemize}

\end{itemize}

Note that for these comparisons, all MOEA configurations were tested against randomly generated puzzles
for concision.

Tables \ref{outline1}, \ref{outline2}, \ref{outline3}, \ref{outline4}, and \ref{outline5}
outline the type of configurations compared, the figures associated
with each MOEA, and the table containing statistical analysis results for each comparison, where 
applicable. Note: not all MOEA configuration subfitness graphs are included in this report for
clarity.

TODO: ADD FIGURES? REMOVE FIGURE LABELS FROM OUTLINE TABLES?

Table \ref{outline1} outlines the comparison made between an MOEA configured with 
fitness proportional parent and survival selection versus tournament
parent and survival selection, both using the plus survival strategy. Table 
\ref{fitness_tourny_plus} illustrates an f-test being applied resulting in equal variance assumption
followed by a t-test assuming equal variances. The result of the t-test was that 
the MOEA configuration employing tournament for both parent and survival selection was superior, for
the given test cases, to the MOEA configured to use fitness proportional selection for both the 
parent and survival selection. The fact that the binary tournament configuration produced the better
performing algorithm means that a lower selection pressure drives up algorithm performance, as 
a binary tournament applies less selection pressure than fitness proportional selection.

The next comparison, outlined in Table \ref{outline2} and statistically analyzed in Table
\ref{fitness_tourny_comma}, compared a fitness proportional parent and survival 
selection configuration with a tournament parent and survival selection configuration, with both
configurations using a comma survival strategy. This comparison again involved an f-test which
yielded that equal variances could be assumed, followed by a t-test assuming equal variances. The
t-test yielded that neither configuration was statistically better for the given test cases. Unlike
for the plus survival strategy, the comma survival strategy does not seem to be as sensitive to
selection pressure brought about by varying survival and parent selection methods. This may be 
due to the fact that the comma survival strategy automatically removes all parent genotypes from the
population after each subsequent generation, only allowing children genotypes to comprise each
new generation, ensuring that stale, yet optimally fit, genotypes that have existed for generations
are removed for new genotypes to take their place.

The third comparison is outlined in Table \ref{outline3} and is statistically analyzed in Table
\ref{fitness_tourny_tourny_fitness_plus}. This experiment yielded that the fitness proportional 
parent selection and tournament survival selection configuration outperformed the 
tournament parent selection and fitness proportional survival selection configuration. With 
the statistical analysis of this comparison, unequal variances were assumed by the f-test and the
t-test yielded the results mentioned above. This finding showed that higher selection pressure
for parent selection (fitness proportional selection) was more effective than higher selection
pressure for survival selection.


\section{BONUS \#1: Impact of Increasing Number of Objectives on Number of Non-Domination and MOEA Performance}

TODO: need to create config files and run test for this


\section{Appendix: Figures and Tables}


% Validity Enforced plus Uniform Random Init. Figures

\begin{figure}
    \addgraphic{website_puzzle/website_puzzle_lit_cell_ratio}
    \fitnessplotcaption{Lit Cell Ratio Subfitness, Validity Enforced plus Uniform Random Initialized, Provided Puzzle}
    \label{fig:website_v_ratio}
\end{figure}

\begin{figure}
    \addgraphic{website_puzzle/website_puzzle_bulb_shine_constr}
    \fitnessplotcaption{Bulb Shine Constraint Subfitness, Validity Enforced plus Uniform Random Initialized, Provided Puzzle}
    \label{fig:website_v_shine}
\end{figure}

\begin{figure}
    \addgraphic{website_puzzle/website_puzzle_black_cell_constr}
    \fitnessplotcaption{Black Cell Constraint Subfitness, Validity Enforced plus Uniform Random Initialized, Provided Puzzle}
    \label{fig:website_v_black}
\end{figure}

\begin{figure}
    \addgraphic{random_gen/random_gen_lit_cell_ratio}
    \fitnessplotcaption{Lit Cell Ratio Subfitness, Validity Enforced plus Uniform Random Initialized, Randomly Generated Puzzle}
    \label{fig:random_gen_v_ratio}
\end{figure}

\begin{figure}
    \addgraphic{random_gen/random_gen_bulb_shine_constr}
    \fitnessplotcaption{Bulb Shine Constraint Subfitness, Validity Enforced plus Uniform Random Initialized, Randomly Generated Puzzle}
    \label{fig:random_gen_v_shine}
\end{figure}

\begin{figure}
    \addgraphic{random_gen/random_gen_black_cell_constr}
    \fitnessplotcaption{Black Cell Constraint Subfitness, Validity Enforced plus Uniform Random Initialized, Randomly Generated Puzzle}
    \label{fig:random_gen_v_black}
\end{figure}


% Uniform Random Init. Figures

\begin{figure}
    \addgraphic{website_puzzle/website_puzzle_uniform_random_init_lit_cell_ratio}
    \fitnessplotcaption{Lit Cell Ratio Subfitness, Uniform Random Initialized, Provided Puzzle}
    \label{fig:website_u_ratio}
\end{figure}

\begin{figure}
    \addgraphic{website_puzzle/website_puzzle_uniform_random_init_bulb_shine_constr}
    \fitnessplotcaption{Bulb Shine Constraint Subfitness, Uniform Random Initialized, Provided Puzzle}
    \label{fig:website_u_shine}
\end{figure}

\begin{figure}
    \addgraphic{website_puzzle/website_puzzle_uniform_random_init_black_cell_constr}
    \fitnessplotcaption{Black Cell Constraint Subfitness, Uniform Random Initialized, Provided Puzzle}
    \label{fig:website_u_black}
\end{figure}

\begin{figure}
    \addgraphic{random_gen/random_gen_uniform_random_init_lit_cell_ratio}
    \fitnessplotcaption{Lit Cell Ratio Subfitness, Uniform Random Initialized, Randomly Generated Puzzle}
    \label{fig:random_gen_u_ratio}
\end{figure}

\begin{figure}
    \addgraphic{random_gen/random_gen_uniform_random_init_bulb_shine_constr}
    \fitnessplotcaption{Bulb Shine Constraint Subfitness, Uniform Random Initialized, Randomly Generated Puzzle}
    \label{fig:random_gen_u_shine}
\end{figure}

\begin{figure}
    \addgraphic{random_gen/random_gen_uniform_random_init_black_cell_constr}
    \fitnessplotcaption{Black Cell Constraint Subfitness, Uniform Random Initialized, Randomly Generated Puzzle}
    \label{fig:random_gen_u_black}
\end{figure}


% Other EA Config Comparison Figures

%\begin{figure}
%    \addgraphic{random_gen/random_gen}
%    \fitnessplotcaption{}
%    \label{fig:}
%\end{figure}


% Configuration Comparison Outline Tables

\begin{table}[]
\centering
\caption{Fitness Proportional Parent \& Survival Selection Configuration vs Tournament Parent \& Survival 
Selection Configuration, Randomly Generated Puzzle, Comma Survival Strategy, Comparison Info}
\label{outline1}
\resizebox{\textwidth}{!}{%
\begin{tabular}{l|ll}
 & Configuration A & Configuration B \\ \hline
 Parent Selection & Fitness Proportional & Tournament Selection \\ \hline
 Survival Selection & Fitness Proportional & Tournament Selection \\ \hline
 Survival Strategy & Plus & Plus \\ \hline
 \begin{tabular}[c]{@{}l@{}}Subfitness 1 Graph\\ (lit cell ratio)\end{tabular} & Figure N/A & Figure N/A \\ \hline
 \begin{tabular}[c]{@{}l@{}}Subfitness 2 Graph\\ (bulb shine constraint\\ violations)\end{tabular} & Figure N/A & Figure N/A \\ \hline
 \begin{tabular}[c]{@{}l@{}}Subfitness 3 Graph\\ (black cell constraint\\ violations)\end{tabular} & Figure N/A & Figure N/A \\ \hline
 \textbf{Statistical Analysis} & \multicolumn{2}{l}{\textbf{Table \ref{fitness_tourny_plus}}}
 \end{tabular}%
 }
 \end{table}

\begin{table}[]
\centering
\caption{Fitness Proportional Parent \& Survival Selection Configuration vs Tournament Parent \& Survival Selection Configuration, Randomly
Generated Puzzle, Comma Survival Strategy, Comparison Info}
\label{outline2}
\resizebox{\textwidth}{!}{%
\begin{tabular}{l|ll}
 & Configuration A & Configuration B \\ \hline
 Parent Selection & Fitness Proportional & Tournament Selection \\ \hline
 Survival Selection & Fitness Proportional & Tournament Selection \\ \hline
 Survival Strategy & Comma & Comma \\ \hline
 \begin{tabular}[c]{@{}l@{}}Subfitness 1 Graph\\ (lit cell ratio)\end{tabular} & Figure N/A & Figure N/A \\ \hline
 \begin{tabular}[c]{@{}l@{}}Subfitness 2 Graph\\ (bulb shine constraint\\ violations)\end{tabular} & Figure N/A & Figure N/A \\ \hline
 \begin{tabular}[c]{@{}l@{}}Subfitness 3 Graph\\ (black cell constraint\\ violations)\end{tabular} & Figure N/A & Figure N/A \\ \hline
 \textbf{Statistical Analysis} & \multicolumn{2}{l}{\textbf{Table \ref{fitness_tourny_comma}}}
 \end{tabular}%
 }
 \end{table}

\begin{table}[]
\centering
\caption{Fitness Proportional Parent Selection \& Tournament Survival Selection Configuration vs 
Tournament Parent Selection \& Fitness Proportional Survival Selection Configuration, Randomly Generated Puzzle, Plus Survival Strategy, Comparison Info}
\label{outline3}
\resizebox{\textwidth}{!}{%
\begin{tabular}{l|ll}
 & Configuration A & Configuration B \\ \hline
 Parent Selection & Fitness Proportional & Tournament Selection \\ \hline
 Survival Selection & Tournament Selection & Fitness Proportional Selection \\ \hline
 Survival Strategy & Plus & Plus \\ \hline
 \begin{tabular}[c]{@{}l@{}}Subfitness 1 Graph\\ (lit cell ratio)\end{tabular} & Figure N/A & Figure N/A \\ \hline
 \begin{tabular}[c]{@{}l@{}}Subfitness 2 Graph\\ (bulb shine constraint\\ violations)\end{tabular} & Figure N/A & Figure N/A \\ \hline
 \begin{tabular}[c]{@{}l@{}}Subfitness 3 Graph\\ (black cell constraint\\ violations)\end{tabular} & Figure N/A & Figure N/A \\ \hline
 \textbf{Statistical Analysis} & \multicolumn{2}{l}{\textbf{Table \ref{fitness_tourny_tourny_fitness_plus}}}
 \end{tabular}%
 }
 \end{table}

\begin{table}[]
\centering
\caption{Fitness Proportional Parent Selection \&
Tournament Survival Selection Configuration vs Tournament Parent Selection \& Fitness Proportional Survival Selection, 
Randomly Generated Puzzle, Comma Survival Strategy, Comparison Info}
\label{outline4}
\resizebox{\textwidth}{!}{%
\begin{tabular}{l|ll}
 & Configuration A & Configuration B \\ \hline
 Parent Selection & Fitness Proportional & Tournament Selection \\ \hline
 Survival Selection & Tournament Selection & Fitness Proportional \\ \hline
 Survival Strategy & Comma & Comma \\ \hline
 \begin{tabular}[c]{@{}l@{}}Subfitness 1 Graph\\ (lit cell ratio)\end{tabular} & Figure N/A & Figure N/A \\ \hline
 \begin{tabular}[c]{@{}l@{}}Subfitness 2 Graph\\ (bulb shine constraint\\ violations)\end{tabular} & Figure N/A & Figure N/A \\ \hline
 \begin{tabular}[c]{@{}l@{}}Subfitness 3 Graph\\ (black cell constraint\\ violations)\end{tabular} & Figure N/A & Figure N/A \\ \hline
 \textbf{Statistical Analysis} & \multicolumn{2}{l}{\textbf{Table \ref{fitness_tourny_tourny_fitness_comma}}}
 \end{tabular}%
 }
 \end{table}

\begin{table}[]
\centering
\caption{Tournament Parent \& Survival Selection (Comma Survival Strategy) vs Tournament Parent \& 
Survival Selection (Plus Selection Strategy), Randomly Generated Puzzle, Comparison Info}
\label{outline5}
\resizebox{\textwidth}{!}{%
\begin{tabular}{l|ll}
 & Configuration A & Configuration B \\ \hline
 Parent Selection & Tournament Selection & Tournament Selection \\ \hline
 Survival Selection & Tournament Selection & Tournament Selection \\ \hline
 Survival Strategy & Comma & Plus \\ \hline
 \begin{tabular}[c]{@{}l@{}}Subfitness 1 Graph\\ (lit cell ratio)\end{tabular} & Figure N/A & Figure N/A \\ \hline
 \begin{tabular}[c]{@{}l@{}}Subfitness 2 Graph\\ (bulb shine constraint\\ violations)\end{tabular} & Figure N/A & Figure N/A \\ \hline
 \begin{tabular}[c]{@{}l@{}}Subfitness 3 Graph\\ (black cell constraint\\ violations)\end{tabular} & Figure N/A & Figure N/A \\ \hline
 \textbf{Statistical Analysis} & \multicolumn{2}{l}{\textbf{Table \ref{tourny_comma_tourny_plus}}}
 \end{tabular}%
 }
 \end{table}


% Statistical Analysis Tables

\begin{table}[] 
\tablecaption{Uniform Random and Validity Enforced Uniform Random Initialized, Randomly Generated Puzzle, EA configurations}        
\label{init_random}                 
\resizebox{\textwidth}{!}{%        
\begin{tabular}{|l|l|l|}           
\hline               
  & random\_gen & random\_gen\_uniform\_random\_init  \\ \hline 
 mean & 1.10325284795678 & 1.0764116503308314 \\ \hline 
 variance & 0.04970344216568983 & 0.04355857958196582 \\ \hline 
 standard deviation & 0.22294268807406498 & 0.20870692269775296 \\ \hline 
 observations & 30 & 30 \\ \hline 
 df & 29 & 29 \\ \hline 
 F & 1.141071234248146 &   \\ \hline 
 F critical & 0.5373999648406917 &   \\ \hline 
 Unequal variances assumed &  &    \\ \hline 
  &  &     \\ \hline 
  observations & 30 &  \\ \hline 
  df & 31 &  \\ \hline 
  t Stat & 0.47331304656977363 &  \\ \hline 
  P two-tail & 0.6377741699825987 &  \\ \hline 
  t Critical two-tail & 2.0395 &  \\ \hline 
  Nether random\_gen\_uniform\_random\_init nor & & \\
 random\_gen is statistically better &  &   \\ \hline 
  \end{tabular}%     
}                    
\end{table}

\begin{table}[] 
\tablecaption{Uniform Random and Validity Enforced Uniform Random Initialized, Provided Puzzle, EA configurations}        
\label{init_provided}                 
\resizebox{\textwidth}{!}{%        
\begin{tabular}{|l|l|l|}           
\hline               
  & website\_puzzle & website\_puzzle\_uniform\_random\_init  \\ \hline 
 mean & 0.8031737242867948 & 0.7959489651519909 \\ \hline 
 variance & 0.000603666878279215 & 0.00058947613524421 \\ \hline 
 standard deviation & 0.024569633254878164 & 0.02427912962287178 \\ \hline 
 observations & 30 & 30 \\ \hline 
 df & 29 & 29 \\ \hline 
 F & 1.02407348183676 &   \\ \hline 
 F critical & 0.5373999648406917 &   \\ \hline 
 Unequal variances assumed &  &    \\ \hline 
  &  &     \\ \hline 
  observations & 30 &  \\ \hline 
  df & 31 &  \\ \hline 
  t Stat & 1.1263571505364551 &  \\ \hline 
  P two-tail & 0.26465388827990055 &  \\ \hline 
  t Critical two-tail & 2.0395 &  \\ \hline 
  Nether website\_puzzle\_uniform\_random\_init nor & & \\
 website\_puzzle is statistically better &  &   \\ \hline 
  \end{tabular}%     
}                    
\end{table}

\begin{table}[] 
\tablecaption{Fitness Proportional Parent \& Survival Selection Configuration vs Tournament Parent \& Survival Selection Configuration, Randomly Generated Puzzle, Plus Survival Strategy}        
\label{fitness_tourny_plus}                 
\resizebox{\textwidth}{!}{%        
\begin{tabular}{|l|l|l|}           
\hline               
  & random\_gen\_\_fitness\_proportional\_parent\_\_fitness\_proportional\_survival\_\_plus & random\_gen\_\_tournament\_parent\_\_tournament\_survival\_\_plus  \\ \hline 
 mean & 0.9712902683331008 & 1.1621500278718004 \\ \hline 
 variance & 0.05379675879756281 & 0.06133986851049603 \\ \hline 
 standard deviation & 0.23194128308165152 & 0.2476688686744784 \\ \hline 
 observations & 30 & 30 \\ \hline 
 df & 29 & 29 \\ \hline 
 F & 0.8770276184787956 &   \\ \hline 
 F critical & 0.5373999648406917 &   \\ \hline 
 Equal variances assumed &  &    \\ \hline 
  &  &     \\ \hline 
  observations & 30 &  \\ \hline 
  df & 58 &  \\ \hline 
  t Stat & -3.0290512365769584 &  \\ \hline 
  P two-tail & 0.0036590951523001046 &  \\ \hline 
  t Critical two-tail & 2.0017 &  \\ \hline 
  random\_gen\_\_tournament\_parent\_\_tournament\_survival\_\_plus is statistically better than random\_gen\_\_fitness\_proportional\_parent\_\_fitness\_proportional\_survival\_\_plus &  &   \\ \hline 
  \end{tabular}%     
}                    
\end{table}

\begin{table}[] 
\tablecaption{Fitness Proportional Parent \& Survival Selection Configuration vs Tournament Parent \& Survival Selection Configuration, Randomly Generated Puzzle, Comma Survival Strategy}        
\label{fitness_tourny_comma}                 
\resizebox{\textwidth}{!}{%        
\begin{tabular}{|l|l|l|}           
\hline               
  & random\_gen\_\_fitness\_proportional\_parent\_\_fitness\_proportional\_survival\_\_comma & random\_gen\_\_tournament\_parent\_\_tournament\_survival\_\_comma  \\ \hline 
 mean & 1.0292821828716079 & 1.053450949371725 \\ \hline 
 variance & 0.036008620889199365 & 0.026268301489784093 \\ \hline 
 standard deviation & 0.18975937628796993 & 0.16207498724289351 \\ \hline 
 observations & 30 & 30 \\ \hline 
 df & 29 & 29 \\ \hline 
 F & 1.3708012641473353 &   \\ \hline 
 F critical & 0.5373999648406917 &   \\ \hline 
 Equal variances assumed &  &    \\ \hline 
  &  &     \\ \hline 
  observations & 30 &  \\ \hline 
  df & 58 &  \\ \hline 
  t Stat & -0.5215427512917932 &  \\ \hline 
  P two-tail & 0.6039746747540546 &  \\ \hline 
  t Critical two-tail & 2.0017 &  \\ \hline 
  Nether random\_gen\_\_tournament\_parent\_\_tournament\_survival\_\_comma nor & & \\
 random\_gen\_\_fitness\_proportional\_parent\_\_fitness\_proportional\_survival\_\_comma is statistically better &  &   \\ \hline 
  \end{tabular}%     
}                    
\end{table}

\begin{table}[] 
\tablecaption{Fitness Proportional Parent Selection \& Tournament Survival Selection Configuration vs Tournament Parent Selection \& Fitness Proportional Survival Selection Configuration, Randomly Generated Puzzle, Plus Survival Strategy}        
\label{fitness_tourny_tourny_fitness_plus}                 
\resizebox{\textwidth}{!}{%        
\begin{tabular}{|l|l|l|}           
\hline               
  & random\_gen\_\_fitness\_proportional\_parent\_\_tournament\_survival\_\_plus & random\_gen\_\_tournament\_parent\_\_fitness\_proportional\_survival\_\_plus  \\ \hline 
 mean & 1.1900153275106078 & 0.9222749663618831 \\ \hline 
 variance & 0.06049925303163906 & 0.028006442701535593 \\ \hline 
 standard deviation & 0.2459659590911699 & 0.16735125545252294 \\ \hline 
 observations & 30 & 30 \\ \hline 
 df & 29 & 29 \\ \hline 
 F & 2.160190556022379 &   \\ \hline 
 F critical & 0.5373999648406917 &   \\ \hline 
 Unequal variances assumed &  &    \\ \hline 
  &  &     \\ \hline 
  observations & 30 &  \\ \hline 
  df & 31 &  \\ \hline 
  t Stat & 4.846489025268326 &  \\ \hline 
  P two-tail & 1.2117965929260821e-05 &  \\ \hline 
  t Critical two-tail & 2.0395 &  \\ \hline 
  random\_gen\_\_fitness\_proportional\_parent\_\_tournament\_survival\_\_plus is statistically better than random\_gen\_\_tournament\_parent\_\_fitness\_proportional\_survival\_\_plus &  &   \\ \hline 
  \end{tabular}%     
}                    
\end{table}

\begin{table}[] 
\tablecaption{Fitness Proportional Parent Selection \& Tournament Survival Selection Configuration vs Tournament Parent Selection \& Fitness Proportional Survival Selection, Randomly Generated Puzzle, Comma Survival Strategy}        
\label{fitness_tourny_tourny_fitness_comma}                 
\resizebox{\textwidth}{!}{%        
\begin{tabular}{|l|l|l|}           
\hline               
  & random\_gen\_\_fitness\_proportional\_parent\_\_tournament\_survival\_\_comma & random\_gen\_\_tournament\_parent\_\_fitness\_proportional\_survival\_\_comma  \\ \hline 
 mean & 1.1186544987668476 & 1.0557988630231476 \\ \hline 
 variance & 0.04925392618449863 & 0.027554956293447713 \\ \hline 
 standard deviation & 0.22193225584510834 & 0.16599685627579733 \\ \hline 
 observations & 30 & 30 \\ \hline 
 df & 29 & 29 \\ \hline 
 F & 1.787479742662873 &   \\ \hline 
 F critical & 0.5373999648406917 &   \\ \hline 
 Unequal variances assumed &  &    \\ \hline 
  &  &     \\ \hline 
  observations & 30 &  \\ \hline 
  df & 31 &  \\ \hline 
  t Stat & 1.221342582229864 &  \\ \hline 
  P two-tail & 0.22729007385576178 &  \\ \hline 
  t Critical two-tail & 2.0395 &  \\ \hline 
  Nether random\_gen\_\_tournament\_parent\_\_fitness\_proportional\_survival\_\_comma nor & & \\
 random\_gen\_\_fitness\_proportional\_parent\_\_tournament\_survival\_\_comma is statistically better &  &   \\ \hline 
  \end{tabular}%     
}                    
\end{table}

\begin{table}[] 
\tablecaption{Tournament Parent \& Survival Selection (Comma Survival Strategy) vs Tournament Parent \& Survival Selection (Plus Selection Strategy), Randomly Generated Puzzle}        
\label{tourny_comma_tourny_plus}                 
\resizebox{\textwidth}{!}{%        
\begin{tabular}{|l|l|l|}           
\hline               
  & random\_gen\_\_tournament\_parent\_\_tournament\_survival\_\_comma & random\_gen\_\_tournament\_parent\_\_tournament\_survival\_\_plus  \\ \hline 
 mean & 1.053450949371725 & 1.1621500278718004 \\ \hline 
 variance & 0.026268301489784093 & 0.06133986851049603 \\ \hline 
 standard deviation & 0.16207498724289351 & 0.2476688686744784 \\ \hline 
 observations & 30 & 30 \\ \hline 
 df & 29 & 29 \\ \hline 
 F & 0.4282418943446097 &   \\ \hline 
 F critical & 0.5373999648406917 &   \\ \hline 
 Unequal variances assumed &  &    \\ \hline 
  &  &     \\ \hline 
  observations & 30 &  \\ \hline 
  df & 31 &  \\ \hline 
  t Stat & -1.9776642228780876 &  \\ \hline 
  P two-tail & 0.05349296004844127 &  \\ \hline 
  t Critical two-tail & 2.0395 &  \\ \hline 
  Nether random\_gen\_\_tournament\_parent\_\_tournament\_survival\_\_plus nor & & \\
 random\_gen\_\_tournament\_parent\_\_tournament\_survival\_\_comma is statistically better &  &   \\ \hline 
  \end{tabular}%     
}                    
\end{table}

\end{document}

