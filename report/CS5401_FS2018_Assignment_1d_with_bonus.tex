\documentclass[11pt]{article}
\usepackage[left=3cm, right=3cm, top=2cm]{geometry}
\usepackage{graphicx}

% Prevent heading numbers
\setcounter{secnumdepth}{0}

% Command definitions 
\newcommand{\fitnessplotcaption}[1]{\caption{Evaluations versus Average Local Fitness and Evaluations versus 
    Local Best Fitness for the \textbf{{#1}}, Averaged Over All Runs}}

\newcommand{\addgraphic}[1]{\centerline{\includegraphics[scale=0.6]{report/figures/{#1}.png}}}

\newcommand{\tablecaption}[1]{\caption{Statistical Analysis performed on the {#1}, EA configurations}}

\title{CS5401 FS2018 Assignment 1d}
\author{William  Lorey \\ wwlytc@mst.edu}
\date{}

\begin{document}

\maketitle

\tableofcontents

\section{Introduction}

Assignment 1d involved implementing a Multi-Objective Evolutionary Algorithm (MOEA)
to more effectively solve Light Up puzzles by balancing the fulfillment of three
objectives: 

\begin{enumerate}

\item maximize the number of cells lit up (represented in this implementation
as a ratio of lit cells to the total number of white cells)

\item minimize the number of bulbs shining on each other

\item minimize the number of black cell adjacency constraint violations

\end{enumerate}

For BONUS \#1, a fourth objective was added, namely minimizing the number of bulbs
placed on the board.

This report outlines this solution's particular
implementation of a MOEA, the impact of initialization strategies on the MOEA's 
performance, a comparison between parent selection, survival strategy, and survival
selection strategies on MOEA performance, as well as the impact of increasing the
number of objectives on non-domination and MOEA performance (BONUS \#1).

\section{MOEA Overview}

The MOEA implemented in this assignment is based on the NSGA-II algorithm. It begins,
similar to a standard evolutionary algorithm, by creating an initial population using
either uniform random or validity enforced plus uniform random initialization, the 
settings for which are specified in the algorithm configuration file. That population
is evaluated and the subfitnesses are determined and assigned to each individual 
in the population.

The population is then evaluated on the basis of non-domination. A list of Pareto fronts is created
from the initial population where all genotypes in a given front are not dominated
by any other genotypes in that front while genotypes in higher level fronts are dominated
by genotypes in lower level fronts. The 'best' genotypes, those in the best level
of non-domination, are assigned to level number one. Subsequent levels increase in
increments of one for other levels of non-domination.

The fitness of each genotype is then set to its level in the list of Pareto fronts, with
individuals exhibiting a smaller fitness (level number) are more fit. A binary tournament selection 
is performed to choose breeding parents. Then offspring are created using an n-point crossover
recombination (with n determined in the configuration file). Following that, mutation is performed,
completing the child population.

For the standard NSGA-II configuration (exhibited in the deliverables configuration folder),
the plus survival strategy is exhibited, combining the children and parent populations into 
one large population from which to choose the new population. Individuals are then selected for 
survival using a binary tournament selection and the process is repeated using the new population
until the end of the experiment.


\section{Impact of Initialization on MOEA Performance}


\section{Comparison of Parent Selection, Survival Strategy, and Survival Selection Strategies}


\section{BONUS \#1: Impact of Increasing Number of Objectives on Number of Non-Domination and MOEA Performance}

TODO: need to create config files and run test for this

% \begin{figure}
%     \addgraphic{website_puzzle_validity_enforced_graph}
%     \fitnessplotcaption{Penalty Function EA with the Validity Enforced, Provided Puzzle}
%     \label{fig:website_puzzle_validity_enforced_graph}
% \end{figure}

\end{document}
